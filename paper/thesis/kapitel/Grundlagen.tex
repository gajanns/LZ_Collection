\chapter{Grundlagen}

Zunächst stellen wir die verwendete Terminologie und relevante Konzepte bzw. Phänomene dar.

\section{Kompression} \label{comp}

\subsection{Verlustfreie Kompression}
Der Prozess der Kompression überführt eine Repräsentation einer finiten Datenmenge in eine möglichst kompaktere Form. Eine verlustfreie Kompression ist gegeben, falls die Abbildung
zwischen der ursprünglichen und komprimierten Datenmenge bijektiv ist. Die Korrektheit einer verlustfreien Kompression kann daher durch die Angabe einer Dekompressionsfunktion nachgewiesen werden.
Ist diese Vorraussetzung nicht gegeben, so handelt es sich um eine verlustbehaftete Kompression, da eine Rekonstruktion der ursprünglichen Datenmenge nicht garantiert werden kann.

\subsection{Eingabe}
Unsere Eingabe sei durch eine $n$-elementige Zeichenfolge $S=e_1...e_n$ über dem numerischen Alphabet $\Sigma$ mit $e_i\in \Sigma$ $\forall i=1,...,n$ gegeben. Für jede
beliebige Zeichenfolge $S$ wird mit $|S|$ dessen Länge $n$ bezeichnet. Der Ausdruck $S[i..j]\in \Sigma^{j-i+1}$ mit $1\leq i\leq j\leq n$ beschreibt die Teilfolge $e_i...e_j$ , wobei im Falle, 
dass $i=j$ ist, das einzelne Zeichen $e_i$ referenziert wird. Alternativ kann ein einzelnes Zeichen $e_i$ auch durch $S[i]$ referenziert werden. Eine Teilfolge der Form 
$S[1..k]$ mit $k\leq n$ wird als Präfix von $S$ bezeichnet.