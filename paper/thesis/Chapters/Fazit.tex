\chapter{Fazit}
\section{Zusammenfassung und Einordnung}
Im Rahmen dieser Arbeit haben wir die erste Phase eines approximativen LZ77- Algorithmus, der in erster Linie auf die Minimierung
seines Speicherbedarfs abzielt, auch in seiner Laufzeit optimiert. Hierbei spielte neben praktischen Optimierungen insbesondere eine 
parallele Ausführung eine entscheidende Rolle. Die Parallelisierung des Algorithmus zeigte eine deutliche Beschleunigung der Laufzeit ohne
die Qualität der Kompression zu beeinträchtigen. Es ist jedoch zu beachten, dass wir mit einer oberen Schranke für den Grad der Parallelisierung
konfrontiert wurden, die mit hoher Wahrscheinlichkeit durch die begrenzte Bandbreite des Speichersystems verursacht wurde. Weiterhin
stellen die optionalen Optimierungen einen TradeOff zwischen Metriken des Algorithmus dar, welcher je nach Eingabe und der Anforderungen
abgewägt werden muss.

\section{Ideen für die Zukunft}
Die beschriebenen Schwächen bzw. Grenzen des implementierten Algorithmus bieten Raum für zukünftige Verbesserungen. So können weitergehende
Optimierung der Parallelisierung, die auf einer hardwarenahen Steuerung der Speicherzugriffe basieren, die Grenzen der Bandbreite des Speichersystems
besser ausnutzen. Im Laufe der Ausarbeitung dieser Algorithmus wurden alternative Techniken und Bibliotheken getestet, die jedoch zum Zeitpunkt
der Finalisierung dieser Arbeit nicht zufriedenstellend optimiert waren. Beispielsweise könnte die Verwendung eines Bloom-Filters \cite{bloom} das Volumen
der Suchoperationen von Referenzen reduzieren. Weitehrin wäre eine Paralleliserung der weiteren zwei Phasen des approximativen LZ77-Kompressionsalgorithmus
eine sinnvolle Erweiterung.